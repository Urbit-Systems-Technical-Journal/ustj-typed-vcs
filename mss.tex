\documentclass[twoside]{article}

\usepackage{ustj}

\addbibresource{mss.bib}

\newcommand{\authorname}{N. E. Davis}
\newcommand{\authorpatp}{\patp{lagrev-nocfep}}
\newcommand{\affiliation}{Urbit Foundation}

%  Make first page footer:
\fancypagestyle{firststyle}{%
\fancyhf{}% Clear header/footer
\fancyhead{}
\fancyfoot[L]{{\footnotesize
              %% We toggle between these:
              % Manuscript submitted for review.\\
              {\it Urbit Systems Technical Journal} I:2 (2024):  33–48. \\
              ~ \\
              Address author correspondence to \authorpatp.
              }}
}
%  Arrange subsequent pages:
\fancyhf{}
\fancyhead[LE]{{\urbitfont Urbit Systems Technical Journal}}
\fancyhead[RO]{Typed Revision Control}
\fancyfoot[LE,RO]{\thepage}

%%MANUSCRIPT
\title{Typed Revision Control}
\author{\authorname~\authorpatp \\ \affiliation}
\date{}

\begin{document}

\maketitle
\thispagestyle{firststyle}

\begin{abstract}
  
\end{abstract}

% We will adjust page numbering in final editing.
\pagenumbering{arabic}
\setcounter{page}{33}

\tableofcontents

\section{Introduction}

A revision control system\footnote{Also “version control system” ({\sc vcs}); “source control system”.} is responsible for tracking the state and history of assets and asset changes within a particular continuity.  It may also manage the production of assets through a build system, as used for instance by a continuous integration/continuous deployment (CI/CD) workflow.  Trivially, an {\sc rcs} is a logical filesystem:  a database for managing files.  (Conventional {\sc rcs}s are agnostic to the underlying disk hardware, however.\footnote{Compare other logical file systems:  network file systems like Microsoft Server Message Block ({\sc smb}) drives or distributed file systems like the Hadoop Distributed File System ({\sc hdfs}) or the InterPlanetary File System ({\sc ipfs}).})  Computer revision control systems have developed from the 1960s onwards once hard drives permitted files to be stored on tape rather than simply as punchcard files.  Succeeding generations have introduced new approaches and features, converging more or less on the broad functionality today afforded by Git \citep{Torvalds2005} in its elaborated form as a distributed source control system.\footnote{Notably, Git has seemingly failed to fulfill its original promise of being a fully decentralized {\sc rcs}:  most users prefer to use the affordances of centralized Git services such as GitHub or GitLab.}  Much care has been taken to treat the problem of dependencies (e.g. \citet{Ball2015}) and efficient user interfaces (e.g. GitHub, GitLab, etc.).  However, although some modest interest has been expressed for a more complete notion of type in revision control, no current system known to the authors besides Urbit's Clay fully implements a typed revision-controlled filesystem.  In this article, we explore the approach of current {\sc rcs}s and elaborate Clay's contributions to the ongoing task of information management and auditable, reproducible source builds.

\section{Type in Revision Control Systems}

The problem of type in revision control has been of both academic and practical interest for decades \citep{Perry1987}.  Essentially, can file artifacts be organized and classified in such a way as to make them susceptible of comparison across their history?  The simplest arrangement which is commonly made is to designate a file as either plaintext or binary.  A plaintext file can be updated by specifying the character offset and run length to be replaced (an example of a “diff”, or difference between two examples of text; typically these are resolved at the line level rather than character level).  Plaintext files, whether {\sc ascii} or some form of {\sc utf}, are defined by a regular format with well-understood and predictable symbol widths and offsets.  For instance, a little-endian representation of the text \texttt{'clay'} would be reproduced in {\sc ascii} hexadecimal as \texttt{0x7961.6c63} and {\sc ascii} binary as \texttt{0b111.1001.0110.0001.0110.1100.0110.0011}.  (In binary, each character has a leading zero; that is, each entry is seven bits wide plus a 1-bit zero header.)  In contrast, binary files (which is a byword for “everything else”) have arbitrary layout and offset; it may be difficult to determine how and why to compare two revisions of any given file.  Many {\sc rcs}s simply treat all registered data as byte streams.

All modern {\sc rcs}s support plaintext and binary representations of files.  Plaintext diffing at the level of line changes is straightforward; the Hunt–McIlroy algorithm is often employed \citep{Hunt1976}.\footnote{An implementation in Hoon is available at \github{sigilante/diff}.}  However, the most popular revision control systems (such as Subversion and Git) have included only minimal direct support for binary diffs.  Some systems only support storing successive file copies of a binary file; that is, the diff is the entire deletion of the old file and addition of the new file (e.g. {\sc cvs}).  Other systems may support binary file diffs by granular block and offset.  Some {\sc rcs}s support diffing particular binary files using a custom automatic merge tool.  For example, the \texttt{textconv} tool converts each version of the file to a plaintext representation and compares those representations.  (To show a diff for a changes in a {\sc pdf} file, a configured Git instance could compare the text output of \texttt{pdfinfo} in each case as a proxy.  This is generally legible as metadata but does not of itself include sufficient information to reproduce or reverse the intervening changes; the underlying {\sc rcs} still stores the entire binary in its conventional representation.)  In any case, type beyond file extension is not generally supported by {\sc rcs}s; they cannot meaningfully merge files classified as binary.\footnote{As always, there are exceptions to the rules.  The Git {\sc rcs}, for instance, can use a “binary diff” format on exact preimages.  The Subversion {\sc rcs} does provide an elaborated {\sc mime} support system, wherein users can assign each file a {\sc mime} type that Subversion can use in applying and displaying diffs.  However, in practice this only results in the classic text/binary dichotomy with slightly more granularity.  ``Subversion treats all file data as literal byte strings, and files are always stored in the repository in an untranslated state'' (\citeauthor{Collins2016} (\citeyear{Collins2016}), section ``Binary Files and Translation'').}

A general-purpose notion of file type is not isomorphic to conventional type in programming language theory.  For instance, consider a {\sc json} file.  The following are equivalent as {\sc json} artifacts:

\begin{lstlisting}[style=listingcode]
{
  "name": "Alice",
  "age": 30
}
\end{lstlisting}

\begin{lstlisting}[style=listingcode]
{
  "age": 30,
  "name": "Alice"
}
\end{lstlisting}

\begin{lstlisting}[style=listingcode]
{ "name": "Alice", "age": 30, }
\end{lstlisting}

Indeed, since {\sc json} is structurally agnostic to syntactic whitespace, an infinite number of equivalent files could be constructed.  Clearly as plaintext representations, each of these differ, yet they are fully equivalent as {\sc json} artifacts.  An {\sc rcs} which can recognize a {\sc json} input can meaningfully store them as equivalent, while an {\sc rcs} that makes only the trivial distinction between plaintext and binary must consider them all to be different—nonsemantic changes in structure may trigger spurious diffs in the file's history.  Furthermore, a {\sc json}-aware {\sc rcs} could meaningfully merge these files whereas a plaintext/binary {\sc rcs} could not.  Finally, a {\sc json}-aware {\sc rcs} could yield a different plaintext output when asked to produce the file, as it could normalize the whitespace to a canonical form.  This may be considered by other systems to be an ``incorrect'' round trip.

Thus we must consider two intersecting notions of file type:
\begin{enumerate}
  \item The file type as understood by the system, typically based on bitwise reproducibility (e.g. plaintext, binary).
  \item The file type as understood conceptually, typically based on structural relationships (e.g. {\sc json}, {\sc xml}/{\sc html}, Word \texttt{docx}, etc.).
\end{enumerate}

\noindent
The employment of an intermediate representation allows us to track changes to files and manage their history—state—more effectively.  Consider the popular JavaScript framework React \citep{React}, aspects of which were developed to resolve difficulties with slow repainting in browsers (for a popular explanation, see \citeauthor{Baer2018} (\citeyear{Baer2018}), chapter 1).  React's virtual Document Object Model ({\sc dom}) system acts as an intermediate representation between the actual {\sc dom} and the developer's code.  When a developer updates the virtual {\sc dom}, the framework compares the new virtual {\sc dom} with the previous virtual {\sc dom} and updates the real {\sc dom} only if necessary.  This allows React to optimize the rendering process and improve performance.  It also means that we can consider the structural representation of the {\sc dom} object as an intermediate representation, and the production and application of the diff as a type-aware {\sc rcs} operation.  Higher-level interfaces like Microsoft Word's “Track Changes” feature \citep{MS2024} similarly resolves diffs across text and formatting within a document, an artifact processed from a file into a user-friendly representation.  While the “Track Changes” user interface is optimized for interactivity, conceptually it is quite similar to conventional merge tools from {\sc rcs}s, and de facto resolves the same sorts of diffs for its particular file type.\footnote{Indeed, one can imagine a Word-aware command-line {\sc rcs} that could meaningfully merge changes to a Word document, rather than simply storing successive copies of the file.  The OpenDocument file form (\texttt{odt}, \texttt{ods}, \texttt{odg}, etc.) consists of \textsc{ZIP}-compressed \textsc{xml} files, which would be highly amenable to type-aware merge tools at the command line.}  Furthermore, the ``Source Code in Database'' (\textsc{scid}) approach parses all code into a database as an intermediate representation; this shares some features with a hypothetical type-aware \textsc{rcs}.  Thus a significant part of managing typed file data is deciding what the input, output, and diffs should look like, and the remainder is systematic application of these rules to the data.

Given this frame, consider \citeauthor{Perry1987}'s taxonomy of version control for module interfaces:

\begin{enumerate}
  \item  No version control.  Files are simply replaced.
  \item  Basic version control.  Files are versioned with a number, but no diff is applied.
  \item  Strongly-typed version control.  Files are resolved at a structural level, such as procedures or arrays, and diffs are applied at that level.
\end{enumerate}

\noindent
In this scheme, contemporary Git-style {\sc rcs} is a ``weakly-typed'' {\sc rcs}:  the system is aware of versions and diffs, but typically not below the file or line level.\footnote{Oddly, Perry did not call out weak typing in his enumeration, although he did address it elsewhere in the article.}  Perry was concerned with the behavior and representation of semantic objects, rather than files or data:

\begin{quote}
  Version equivalence in this type of version control mechanism is defined in terms of syntactic equivalence:  data objects are equivalent if their types or structures are equivalent; operations and modules are equivalent if their signatures are equivalent.  \citep{Perry1987}
\end{quote}

\noindent
Strongly-typed version control requires coupling to the build system, since the system must be aware of the structure of the data in order to apply diffs.  (Indeed, \citeauthor{Blackman2024} (\citeyear{Blackman2024}), pp. 6–7, in \ssc{ustj} vol. 1 iss. 1, pp. 1—46, discussed one solution to the linking problem which is well-suited to Urbit's global namespace.)


\section{Clay}

\subsection{Revision Control and Marks}

Urbit provides the Clay vane at \lstinline[style=inlinecode]{/sys/vane/clay} to manage Unix-style files and source builds (cf. \citet{Yarvin2016}).  Clay acts as both the Unix-like filesystem and the source control vane.  While Hoon itself (\lstinline[style=inlinecode]{/sys/hoon}) is responsible to compile a source noun into a Hoon abstract syntax tree ({\sc ast}) and ultimately into executable Nock, the actual construction of source input from files is delegated to the \lstinline[style=inlinecode]{++ford} arm of Clay.  This build process includes compiling the correct import files (such as libraries and shared structure files).  Clay is also tightly coupled to the userspace vane Gall since system updates and agent updates both occur via Clay.\footnote{We omit discussion of several other notable properties of Clay as a single-level store:  referential transparency, global addressability, and event-level persistence, for instance.}

To motivate how Clay functions as a revision control filesystem, we need to briefly examine some major data structures utilized by Clay.  At the highest level, files are organized into “desks”, described in previous literature as being analogous to Git branches but really more like Git repositories.  A desk is a self-contained continuity of files; its state is the result of a history of commits.  Desks may not include links to off-desk resources.  Each desk must expose a few files at definite paths; primarily \lstinline[style=inlinecode]{/mar} for the marks used to read the desk contents.\footnote{While Arvo, the vanes, and userspace tools look at other definite locations, in principle one could construct a desk of arbitrary {\sc url}-safe paths as long as they are legible to Clay's build process via \lstinline[style=inlinecode]{/mar}.}

Ultimately, Arvo and Clay deal in nouns.  A noun is a binary tree of unsigned integers.  These may be evaluated as Nock formulas or manipulated as data.  A noun is formally agnostic to the underlying hardware and is almost always accessed via some representation path (i.e. as phenomenon).  A noun serves as a universally legible intermediate representation for all data and code.

Urbit has an immutable data model; the system state of Arvo is the unique result of the events in its history, stored as its event log.  If a noun changes, it is the result of definite discrete changes to the system state.  This yields very nice properties of referential transparency and has some ramifications in library management; see \citet{Blackman2024a} for more details.

Typically, a noun is altered either directly via subject-modi\-fying code (\lstinline[style=inlinecode]{%~} censig, \lstinline[style=inlinecode]{=.} tisdot) or via a mark transformation, on which more later.

While conventional file systems denote file type primarily by file suffix or a cursory search using “magic tests” on the file header, Urbit instead stores all data as a noun accessed via a mark.  A mark is essentially a representation rule for nouns.  Marks permit nouns to be stored with more granularity than the text/binary dichotomy facilitates, since details of conversion are stored as part of the mark.
% https://unix.stackexchange.com/questions/207276/how-are-file-types-known-if-not-from-file-suffix

A mark may be considered an executable {\sc mime} type.\footnote{A media type (formerly {\sc mime} type) is a tag identifier for the intended data format of a particular data entry. We will use “{\sc mime} type” as this terminology remains in common use.}  A mark essentially describes how to map a data representation to a noun, and from a noun back out to a data representation.  This conversion may be trivial (e.g. the binary storage of an audio file) or sophisticated (from a text stream to a linked-list {\sc utf} representation).  Data are accessed via a particular mark, and the appropriate conversion routine is invoked.\footnote{This is commonly used with {\sc json} and the Eyre {\sc http} server vane, for instance.}  Clay is capable of searching for transition paths between mark representations, and in principle permits conversion between all compatible representations (if there were a path from Markdown \texttt{md} to rich text \texttt{rtf} to plaintext \texttt{txt} in marks, then the conversion is automatically supported by Clay).

\begin{quote}
  A mark is a term (symbol) which is the Urbit equivalent of a {\sc mime} type, if {\sc mime} types were names of typed validation functions.  (\citet{Whitepaper}, p. 51)
\end{quote}

Unix filesystem mounting and committing involves synchronizing a desk's state with a Unix logical representation of the files involved.  The actual file data are imported as \lstinline[style=inlinecode]{%mime}-typed nouns and then converted to their target mark as file type.

Each mark defines several standard arms:

\begin{enumerate}
  \item  \lstinline[style=inlinecode]{++grab} converts from other marks (by arm name) to a given mark.
  \item  \lstinline[style=inlinecode]{++grow} converts from the mark to other marks.
  \item  \lstinline[style=inlinecode]{++grad} defines the diffs applicators for the mark.
\end{enumerate}

\noindent
Marks do not need to be symmetrical or round-trip.  A {\sc json} input may yield an equivalent {\sc json} output but altered by structural whitespace.  Clay is also proactive in searching for possible mark paths—if an immediate conversion is not available, then a mutually available intermediary form may be used.  To implement a custom filetype, all that is required is the production of a suitable mark file.  Conversions from the filesystem require at a minimum a {\sc mime} mark file, which acts as the global intermediary from the host OS.

\subsection{Desks}

A desk is a filesystem continuity,\footnote{See \citet{Blackman2024} in \ssc{ustj} vol. 1 iss. 1, pp. 1—46, section 2.2 for an exposition of the so-called ``theory of a desk''.} and represents the main way that data are synchronized between Urbit instances (ships).  The state of a desk is a history of sequential commits, with each commit after the first having at least one parent commit.  Desks are often analogized to Git repositories.\footnote{Although desks are synchronized to the host OS as folders, this is a merely a convention, and other schemes of noun organization could employ other organizational abstractions.}  Each desk must contain sufficient internal data—especially marks—to produce its own code, given Hoon and the standard library (which are always available).

Since Clay is a global filesystem, it has to maintain information about both local and remote desks.  A local desk's state maintains information about noun data, build and mark cache state, userspace app state, various policies, and synchronization with the host OS filesystem.

\begin{lstlisting}[caption={Clay desk types},
                   style=listingcode]
::  Domestic desk state.
+$  dojo
  $:  qyx=cult              ::  subscribers
      dom=dome              ::  desk state
      per=regs              ::  read perms per path
      pew=regs              ::  write perms per path
      fiz=melt              ::  state for mega merges
  ==
::  Desk state.
+$  dome
  $:  let=aeon              ::  top id
      hit=(map aeon tako)   ::  versions by id
      lab=(map @tas aeon)   ::  labels
      tom=(map tako norm)   ::  tomb policies
      nor=norm              ::  default policy
      mim=(map path mime)   ::  mime cache
      fod=flue              ::  ford cache
      wic=(map weft yoki)   ::  commit-in-waiting
      liv=zest              ::  running agents
      ren=rein              ::  force agents on/off
  ==
::  Support types
+$  mizu  [p=@u q=(map @ud tako) r=rang]
                            ::  new state
+$  rang                    ::  repository
  $+  rang
  $:  hut=(map tako yaki)   ::  changes
      lat=(map lobe page)   ::  data
  ==
\end{lstlisting}

The keystone structure is \texttt{hit}, which connects a given version of the desk (its \texttt{aeon}, which is merely a \texttt{@ud}) to the tree of associated commits.  These are accessed from the ``slush pool'' of known commits via \texttt{hut} in the Clay's \texttt{rang}, which is not associated with a particular local desk.

Much of Clay's machinery is intended to support noun changes propagated as commits, or desk changes.  These may arise from three sources:  a local host OS filesystem sync, a remote desk sync, or an on-Urbit local edit.  A commit is a snapshot of a set of files with a list of parent commits and a timestamp.

Various merge strategies are available to support desk updating.  Given a commit as a timestamped snapshot of files, how should two conflicting histories be reconciled?  For instance, \lstinline[style=inlinecode]{%fine} is a Git-style ``fast-forward'' for use when histories nest but one is ``ahead''.  \lstinline[style=inlinecode]{%meet} combines changes as long as the same file/noun has not been modified.  (Several others are available and are described in the system documentation.)  While Clay has at times supported noun diffs in commits, it currently simply stores the entire file at each commit.\footnote{No technical limitation is in place; this was merely a simplification while Urbit data storage is relatively small, and it is expected to be reverted as larger loom sizes—and thus larger nouns—are supported.}  Urbit developers in practice have not encountered difficulties with desk and commit management due to this limitation at the current time, but resolution using mark diffs will eventually be desirable.

Structurally, Clay tracks desk state as a sequential \lstinline[style=inlinecode]{+$aeon} or \lstinline[style=inlinecode]{@ud}, but files may be arbitrarily accessed using one of several \lstinline[style=inlinecode]{+$case}s:

\begin{lstlisting}[style=listingcode]
+$  case
  $%  [%da p=@da]     ::  %da:  date
      [%tas p=@tas]   ::  %tas: label
      [%ud p=@ud]     ::  %ud:  sequence
      [%uv p=@uv]     ::  %uv:  hash
  ==
\end{lstlisting}

\noindent
A case resolves into the corresponding \lstinline[style=inlinecode]{aeon} via the canonical timestamp of the commit; the label map; or the hash of the corresponding commit.  A userspace system to support detailed commit messages, called Story, is available at \github{urbit/story}.

\subsection{Runtime}

As a vane of Arvo, Clay lives ``on Mars''; that is, it describes a logical filesystem inside of the Urbit ship but does not directly map its nouns to physical hardware.  The particular implementation used to store the total state of Arvo, which contains Clay's state including file data, depends on the runtime.  At the time of writing, the Vere runtime uses the Lightning Memory-Mapped Database Manager ({\sc lmdb}) which is a B-tree representation for a key-value store \citep{LMDB}.  While of practical consequence for lookup time and system behavior, the architecture of Clay to an Urbit developer is independent of the runtime implementation.

The {\sc posix}-compliant host OS supports a filesystem mirror of Clay files organized by desk.  Importantly, while these files can be synced to Clay, they are not the canonical reference for each (which lives in {\sc lmbd}).  Clay maintains a ``sync duct'' with the runtime, which upon receipt of a \lstinline[style=inlinecode]{%dirk} task produces a list of file changes each way and implements the latest changes.\footnote{The relevant code is in \github{urbit/vere} at \lstinline[style=inlinecode]{/vere/io/unix.c}.}

In principle, as we have intimated, an {\sc rcs} may abstract away from the file as the fundamental artifact being tracked.  While Clay does not fully commit to this, in a concession to its {\sc posix}-compliant host OS, it gestures towards the possibility of conceiving of build sources and artifacts as objects (nouns) other than files.  Since Urbit ``unifies'' nouns (that is, it structurally stores only one copy of a noun and maintains references to that shared value), repetition of critical resources across desks does not lead to filesystem bloat.  (See \citet{Blackman2024a} in \ssc{ustj} vol. 1 iss. 1, pp. 75—82, for details.)

\subsection{Building Code}

Clay is also directly responsible for building code.  Formerly, Urbit built code from source files using the Ford vane.  In 2020, the core developers realized that integrating the build system directly into the {\sc rcs} solved a number of problems with updating the system correctly.  Updates to the system should be atomic (complete or fail in one transaction); self-contained (no implicit dependencies or dependencies on previous system states); and ordered (sequenced within the system stack).  By coupling the build system into the {\sc rcs}, the system can be updated in a single transaction, with the build system ensuring that the system is always in a consistent state.  In a typed {\sc rcs}, the compiler or build system must be always at hand to process diffs in artifacts.  Although of relatively minor consequence to userspace application developers, the integration of \lstinline[style=inlinecode]{++ford} into Clay transforms the system closer into an integrated {\sc rcs} for generic nouns.

\subsection{Shortcomings}

As currently implemented, Clay presents several wrinkles in developer ergonomics.  Chief among these is the way that marks, as simple type tags, are underspecified.  The original philosophy of marks centered on network transmission of nouns as tagged data:

\begin{quote}
  Like a {\sc mime} type, the mark is just a label. There is no way to guarantee that the sender and receiver agree on what this label means.  A noun which doesn't normalize to itself is a packet drop.  (\citet{Whitepaper}, pp. 51–52)
\end{quote}

\noindent
In practice, different Clay desks and different Urbit versions can have different mark processors (and thus different behvaior for the same tag).  No simple version system tracks these; there is no global type registry (other than the scry namespace), and any given mark is simply identified by its desk and path.  Developers have expressed a desire for a more robust mark management capability, and solutions have ranged from a global type registry (e.g. that proposed by \href{https://www.archetype.computer/}{Archetype}) to using paths instead of simple tags for marks.  In practice as of writing, marks are disambiguated by supplying the intended mark with the desk; this may lead to multiple different implementations of a mark being present on the same system, although uniquely identified by different Clay paths.

The current architecture of Clay (largely in place by 2016 except for the unification of the former Ford vane into Clay in 2020) leads to three ongoing developer concerns:\footnote{This section benefited from discussion with \patp{tiller-tolbus} about the upcoming ``shrubbery'' project to rework Urbit's userspace.}
% https://urbit.org/blog/toward-a-new-clay
% https://urbit.org/blog/ford-fusion

\begin{enumerate}
  \item  Implementing {\sc rcs} behavior at the file system layer seems to be incorrect for Urbit (that is, too much of a sop to Unix).  trying to do revision control at base layer, but file system is wrong layer (you just want history of actions/tx log rather than full state); see ~migrev codebases for clay state
  \item  Desks seem to be the wrong abstraction in practice.  The theory of the desk doesn't account for actual distribution patterns people would like to use; we want one desk but one spot in file tree is capable of acting as a desk (chroot analysis); overuse of desks makes Urbit not feel as much like a filesystem you can explore
  \item  The application system (userspace) and the build system and filesystem should be co-located.  Currently, Clay and Gall are deeply entangled but must interact via a somewhat awkward message-passing interface.  This leads to a number of ergonomic issues, such as the need to suspend a desk in order to suspend an agent.  Employing paths rather than simple tags would lead to a desire to constrain what can exist at certain paths.  This relates to the definition of files as data structures, and ultimately a noun-maximalist scenario abstracts more towards nouns as database entries rather than ``files''.
\end{enumerate}

One possible future for Clay sees it being restricted to source and build management, rather than expansion to a more fully-featured filesystem.  In this contingency, the userspace management vane may take over conventional file storage, as Gall or a successor.  This may only implement a filesystem interface for the host OS and external applications that ``think'' in files, such as web browsers.

Alternatively, the build system, {\sc rcs}, and application engine fuse into a single userspace vane, which is a possible endpoint of the ``shrubbery'' project.  Mark cores would be replaced with simpler conversion rules permitting only straightforward type casts between nouns as \lstinline[style=inlinecode]{$-(from to)} gates.\footnote{At least two previous models integrating the application layer with the filesystem and build system have been proposed:  \href{https://gist.github.com/arthyn/0476d2f4e9912aa46746c3dd3cd20d15}{Hume} and the \href{https://gist.github.com/arthyn/f805f8294791adccbdb6e0454defb67a}{insect model}.  The insect model in particular surfaced the very Martian concept that ``an agent and a document are the same thing.''}

\section{Conclusion}

In its current instantiation, Clay exhibits some notable characteristics as a typed revision control system.  It is capable of tracking file changes at a structural level, permitting meaningful diffs and merges for files of the same mark.  It is also capable of converting between marks, permitting a more flexible approach to file type than the conventional text/binary dichotomy of Git and other {\sc rcs}s.  However, the future of userspace and code building is still in flux and may see a reworking of the Clay vane to better suit the needs of developers and users.\tombstone{}

\selectlanguage{USenglish}
\printbibliography
\end{document}
